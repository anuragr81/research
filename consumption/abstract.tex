%% LyX 2.0.8.1 created this file.  For more info, see http://www.lyx.org/.
%% Do not edit unless you really know what you are doing.
\documentclass[english]{article}
\usepackage[T1]{fontenc}
\usepackage[latin9]{inputenc}
\usepackage{babel}
\begin{document}
\begin{abstract}
Status competitions amongst workers often explain the income distribution
better than the permanent income or productivity models. The current
study looks for instances of status competitions within communities
in developing countries, focusing on sub-Saharan African countries
where recent large-scale urban migrations and proliferation of industrial
products has occurred amidst social upheavals. The goal of this study
is twofold. First, it comments on changes in status competitions in
response to proliferation of industrial products, changing family
characteristics, regional effects and social interaction levels. Second,
it compares the effect of advertising and marketing on consumption
in the developed world with the emergence of new social scarcities
in the developed world - arguing that the socio-political environment
of the developing world has done little to curtail or encourage status
competitions at local levels. The study uses the framework and terminology
of Fred Hirsch who had discussed the increasing effect of advertising
and ``congestions'' related to positional consumption in the post-war
distribution of resources. The contention of the study is that physical
scarcity in the developing countries does not rule out social scarcity
and that the status competitions and positional consumption - which
are identified by observed differences in prices in the study - must
be considered for a better understanding of urban migration patterns
and consumption pressures.\end{abstract}

\end{document}
