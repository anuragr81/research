%% LyX 2.0.8.1 created this file.  For more info, see http://www.lyx.org/.
%% Do not edit unless you really know what you are doing.
\documentclass[english]{article}
\usepackage[T1]{fontenc}
\usepackage[latin9]{inputenc}
\usepackage{babel}
\begin{document}
\begin{abstract}
Status competitions amongst workers often explain the distribution
of their incomes better than permanent income or productivity models.
The current study looks for instances of status competitions within
communities in the developing countries - focusing on sub-Saharan
Africa where large scale urban migrations and more recent proliferation
of industrial products have occurred amidst social upheavals. The
goal of the study is twofold. First, it comments on changes in status
competitions in response to proliferation of industrial products,
family characteristics, region and social interaction levels. Second,
it compares the effect of advertising and marketing on consumption
in the developed world with the social scarcities in the developed
world - arguing that the socio-political environment of the developing
world has done little to curtail or encourage status competitions
at local levels - whose importance may be growing with bottom-of-pyramid
initiatives. The study uses the terminology of Fred Hirsch - while
discussing the increasing effect of advertising and congestions in
the post-war distribution of resources. The contention of the study
is that physical scarcity doesn't deny social scarcity and that the
``congestion'' for positional goods (which are identified based
on observed differences in prices) must bear a strong relationship
with urban migration patterns.\end{abstract}

\end{document}
