%% LyX 2.1.4 created this file.  For more info, see http://www.lyx.org/.
%% Do not edit unless you really know what you are doing.
\documentclass[english]{article}
\usepackage[T1]{fontenc}
\usepackage[utf8]{luainputenc}
\usepackage{geometry}
\geometry{verbose,tmargin=2.5cm,bmargin=2.5cm,lmargin=2.5cm,rmargin=2.5cm,headheight=2.5cm,headsep=2.5cm,footskip=2.5cm}
\usepackage{array}
\usepackage{calc}
\usepackage{multirow}
\usepackage{amsmath}
\usepackage{lscape}
\usepackage{graphicx}
\PassOptionsToPackage{normalem}{ulem}
\usepackage{ulem}

\makeatletter

%%%%%%%%%%%%%%%%%%%%%%%%%%%%%% LyX specific LaTeX commands.
%% Because html converters don't know tabularnewline
\providecommand{\tabularnewline}{\\}

\makeatother

\usepackage{babel}
\begin{document}

\part{Scarcity and Visible Consumption (contd.)}


\section{Detecting and measuring positional consumption}


\subsection{Measures for Scarcity \cite{HirschSocialLimits}}

Heffetz finds that the degree to which people notice items explains
the corresponding (permanent) income elasticities better. This observation
has provided the basis for inspection of visible consumption in many
studies thereforth. In an environment of inequalities, however, it
is likely that the indvidiuals with perceived higher status notice
items differently from how the lower status individuals might notice
them. The social factors thus relevant for the difference in visible
preferences are sought in the studies on visible consumption in the
developing countries. In India, this is found to be religion and caste
- while in South Africa and United States, race seems to have a dominant
significance. It is also worth noting that the developing countries
may offer a less consumerist agrarian environment overall where expenditure
is more visible than in a relatively individualistic and industrialized
society.

For visibility to bear significance in an environment of severe inequalities
and scarcities, an association with higher income becomes relevant.
Khamis et al \cite{ZahraIndia} perform a slightly more detailed survey
by asking what an individual whose consumption is noticed would do
when her income rises (by 20\%). This quantifies the expectations
from others associates the total expenditure with higher-income. The
items where consumers expect the consumption to rise with increased
income are those that associate with higher income and are declared
``conspicuous'' in the study. 

In a developing economy, the criterion for conspicuous consumption
is clearly not just noticeability any more. Visible consumption may
detail the mechanics of status competitions in a narrow sense where
consumers participate in a market to increase their perceived status
- but it does not provide an adequate picture of conspicuous consumption.
One reason is that markets are underdeveloped in the developing world
and social status is largely yielded through economic classes and
social conditions. The second - probably more significant - reason
is that status signaling does not exist in a society as an inherent
need for visible appeal amongst humans. Instead visible consumption
matters because of an item being associated with a higher status (at
least in the sense which Veblen had talked about in his 19th century
treatise\cite{VeblenLeisureClass}). 

A study of status and scarcity of items therefore goes hand in hand
with the study of conspicuous consumption. Instead of limiting ourselves
to visible consumption as the particular mechanics of status signaling
- where consumers buy items in a common market and (presumably) over-weigh
on items that are more noticeable - we attempt to understand the reasons
behind status-signaling by looking at the differences in patterns
between the richer and poorer sections of society and attempt to understand
how unavailability of items (scarcity) as well as disparities of services
across regions and classes in a society are reflected in both price
and consumption of commodities.

This is not to discount a study of visible consumption or the importance
of a visiblity survey in any way. In fact, visible consumption is
ever more relevant with recent trends in advertisement and consumerism
in the developing world. Even though scarcity is fundamentally more
important than noticeability, we argue that scarcity merely allows
status competitions to develop. The factors that affect status competitions
are indeed beyond scarcity.

With that admission, we decide on the three degrees of scarcity that
we can associate to an item - and define a fundamental assumption
to justify the existence of status competitions arising out of the
scarcity of a commodity. We assume that the richer half of the society
invariably indicates higher status (than lower) and has access to
more facilities (than less). In other words, when individuals are
ranked by permanent income, scarcity is always faced by the lower
half of the society. With this assumption, status competitions would
occur when people with lower income would want to achieve higher status.
The conspicuous items in this view are objects that indicate achieving
what's scarce.

The method with percentile thresholds briefly described in the section
\ref{sub:precentile_threshold_method} measures scarcity on grounds
of i) availability (electricity, food, education etc.) and ii) affordability.
If the item is affordable and not available, it would be classified
as scarce. Severe(1) scarcities - which would be a physical scarcities
in a Hirschian sense - would create minimal status competitions while
under medium scarcity(2), status competition would thrive. For items
that are not scarce(3) at all (i.e. affordable by all and available
to all) would not allow status competitions to develop.


\subsection{Methods to measure Visibility }

The direct way of measuring visibility is to find an evidence of visibility
in the comomdities. Most studies have relied on their own visibility
surveys. In absence of such a survey, one can verify commodities from
public media - e.g. advertisements or social media traffic. These
methods have not been pursued at this stage in the study.


\subsection{\label{sub:precentile_threshold_method}Consumption percentiles}

The differences between amounts spent by the lowest and higher percentile
of spenders of a particular commodity are expected to be higher when
a commodity is a status-good than when it is of common utility. The
illustrations show non-zero log-level expenditure on a few commodities
when the lower (let's say $\theta$) percentile of the consumption
of the commodity is ignored. Ignoring the bottom $\theta$ percentile
of the consumption of a visible commodity is equivalent to treating
the bottom $\theta$ percentile expenditure as non-visible consumption
(If we consider $\theta=10\%$ for electricity, then bottom 10\% percentile
of the consumption on electricity would be consider non-visible and
anything above that level would be considered visible). The plots
of log-expenditures are shown with rising $\theta$ (starting with
the lowest percentile $\theta$ that corresponds to lowest non-zero
log-level of consumption of the commodity).

For a good that is not positional, one expects that the consumers
from lower and higher quantiles of total expenditure (x-axis) would
consume similar amounts of the good (y-axis). For a positional good,
the consumers spending higher expenditure on the good would lean towards
consumers with higher total expenditure. This does not indicate signaling
in any way - but tests only whether a commodity is consumed uniformly
amongst those with lower and higher total expenditure outlays (this
is rather a measure of sarcity of the item than of its visibility).
Choosing different thresholds ($\theta)$ provides a a control on
the degree to which a certain commodity can be included in our conspicuous
consumption basket. Instead of asking whether marriage spending is
visible or not - for example - the test asks if only the richer consumers
can afford a significant expenditure on marriage (while varying the
degree of visibility attached to spending on marriage).

In the data from Tanzania, while top 56\% of consumers show spending
on rice, electricity appears to be a luxury when only top 22 \% of
consumers spend on it. This does not necessitate that a higher consumption
of electricity indicates higher status but a higher threshold for
electricity certainly indicates its physical scarcity which may permit
status competitions.

Not all scarce objects can be indicative of status - we often need
some judgment to decide which products may indicate status signaling.
A survey accomplishes this by ranking all products as viewed by its
respondents. It must also be noted that visibility or positional signaling
of a commodity is hardly orthogonal to major expenditure categories.
The usual arguments of additive utilities cannot hold for conspicuous
consumption. In other words, if walnut turns out to be of visible
significance (ranked high in the consumers' perception of visibility
in the survey) then one can no longer talk about the combined utility
of food and visible items (walnut is both a food item and a visible
good)\textit{. }Detailed microdata thus becomes a necessity for discussing
income elasticities of visible items (\cite{ArrowDasgupta,heffetz_visibility}).

A similar analysis of Consumer Expenditury Survey (CEX) data from
years 2004,2010 and 2014 similarly shows clear differences between
expenditure on jewelry and fruits. It is evident that jewelry is not
popular amidst the relatively poor and that richer consumers spend
a higher portion of their total expenditure on jewelry than on fruits
(curve being steeper for jewelry).

\begin{figure}
\begin{centering}
\includegraphics[scale=0.4]{lsms/loglogrice}
\par\end{centering}

\caption{LSMS Tazanania 2010: Percentiles of nonzero consumption of rice}
\end{figure}


\begin{figure}
\begin{centering}
\includegraphics[scale=0.4]{lsms/loglogelectricity}
\par\end{centering}

\caption{LSMS Tazanania 2010: Percentiles of nonzero consumption of electricity}
\end{figure}


\begin{figure}
\begin{centering}
\includegraphics[scale=0.4]{lsms/loglogfruit}
\par\end{centering}

\caption{LSMS Tazanania 2010: Percentiles of nonzero consumption of fruits }
\end{figure}


\begin{figure}
\begin{centering}
\includegraphics[scale=0.4]{lsms/loglogmarriage}
\par\end{centering}

\caption{LSMS Tazanania 2010: Percentiles of non-zero expenditure on marriage}
\end{figure}


\begin{figure}
\begin{centering}
\includegraphics[scale=0.4]{cex/us_cex_fruits}
\par\end{centering}

\caption{US CEX (2004,2010,2014): Percentiles of non-zero consumption of fruits}
\end{figure}


\begin{figure}
\begin{centering}
\includegraphics[scale=0.4]{cex/us_cex_jewelry}
\par\end{centering}

\caption{US CEX (2004,2010,2014): Percentiles of non-zero consumption of jewelry}
\end{figure}



\section{Analysis of LSMS Data}


\subsection{Steps in preparing LSMS data (2010)}

Following steps were performed before running the regressions on the
household consumption data from LSMS 2010\@.
\begin{enumerate}
\item Read weekly diary data from Section K (a table of items with the quantities
consumed and cost associated with the item for every household).

\begin{enumerate}
\item All items that had no cost assciated with them were \uline{ignored}
(not included in total consumption) 
\item Gift quantities were \uline{ignored} for consumption ( median ratio
of gift to total diary consumption was zero - only 132/3828 households
had this ratio 1\% or higher )
\item Weekly diary data was multiplied by 52 (to \uline{estimate} annual
consumption)

\begin{enumerate}
\item Weekly recall items were also multiplied by 52 (to \uline{estimate}
annual consumption)
\end{enumerate}
\item Monthly recall items were multiplied by 12 (to \uline{estimate}
annual consumption)
\item All expenditure from (c)-(e) above were summed up as total expenditure
\end{enumerate}
\item Obtained Personal Data from Section A,B,C and J files

\begin{enumerate}
\item Section C\_CB was read to obtain market facilitycode and gauge the
accessibility of a market in every district. The closest accessible
market could be either within the district or outside the district
at a given distance. If a market was within the the district or less
than \uline{10 kms away} it was deemed ``accessible''. Urban/rural
classifications based on population density could be inserted at this
stage (population density in not available in LSMS). 
\item Read section B and C files
\item Calculated age of member by subtracting YOB (year-of-birth) from 2010
(survey year) 
\item Read section J for housing data (total house rent, number of primary/secondary
rooms)
\end{enumerate}
\item Obtained income data from Section E (currently ignored for analysis
for it being sparse). Here, the recorded pay frequency was in hours,
days, weeks, months, fortnights, months, quarter, half year or year
- while the mandatory fields corresponding to all of these units were
i) number of hours worked per week ii) number of weeks worked per
month and iii) number of months worked in an year .

\begin{enumerate}
\item When pay was on a per-hour basis, the number of hours worked per week
(provided) was multiplied with the number of weeks worked per month
(provided). This product was then multiplied with the number of months
worked per year (provided) to estimate the annual income.
\item When pay was per-day, a \uline{10 hour working day} was assumed
to obtain the effective number of work-days per week (based on the
number of hours worked per week). This was then mutliplied with the
number of weeks worked per month in the year and then further multiplied
with the number of months worked in an year to obtain the estimated
annual income.
\item When pay was per week, the number of weeks worked per month was multiplied
with the number of months worked per year.
\item When pay was in fortnights, then twice the number of months worked
in an year was used to calculate the total income received over the
year.
\item When pay was per-month, then the multiplication factor was just the
number of months worked per year 
\item When pay was per-quarter, then the effective number of quarters were
inferred from the number of months worked per year (number\_of\_months/3)
and multiplied with the number of months worked per year to obtain
the estimated annual income.
\item For self-employed income, the work-months in an year was similarly
used to compute total income from self-employment in the year
\item All members less than 5 year old were \uline{ignored} from the
income data 
\item For wage workers:

\begin{enumerate}
\item summed up wages into column yearly pay 
\item summed up values under ``other forms of payment''
\item sum up values as secondary of payment (for wage-workers) 
\item only primary job was used to identify the employer type of the individual
\item added other wages from secondary job by summing up yearly-income from
all sources into the yearly income
\end{enumerate}
\end{enumerate}
\item Ignored bad data (outliers)

\begin{enumerate}
\item Ignored 5 households with exceedingly high expenditure on marriage
(more than reported annual income)
\item Ignored households in the income table but with zero income (number
of households with income data thus ignored were under 2\%)
\end{enumerate}
\item Merged all data

\begin{enumerate}
\item Set education expense of houses with education expenses= NA as zero
\item Summed up educational expense and total house rent from personal data
into total expenditure (both weren't a part of diary data)
\item Obtained personids of the house-heads and the following variables
for household-head: education-level, age, years in community, language,
occupation
\item Obtained visible expenditure by summing up expenditure on visible
items
\item Merged all data into one table
\end{enumerate}
\end{enumerate}

\subsection{Claims Tested}


\subsubsection{Effect of occupation}

Income data in LSMS is not available for all the surveyed households.
This may indicate the presence of informal sector in Tanzania. A few
occupations in the survey are neither well defined nor are truly an
indicator of total income. The presence of categories like unpaid-family-work
and of individuals with no-primary-job getting a significant income
from their secondary occupations makes the task of associating the
primary occupation of the household head with her income rather difficult
(i.e. occupation - which is available for all household heads cannot
be used as a proxy of household income - which is not available for
all households in the survey). Grouping the occupations into fewer
categories than in the survey (by putting paid/unpaid family work
and agriculture under the same category for example) allows for the
smoothening of the effect of individual occupations and may serve
as a proxy of socioeconomic classes in the country. Without or without
this grouping, the effect of occupation has been found significant
on the consumption of scarce commodities. The results are shown in
Table \ref{tab:no_instruments_reg1}.


\subsubsection{Effect of Education Level}

One of the claims to be evaluated on the LSMS data is whether education
has a significant effect on visible consumption. If the education
level of NA is considered as none (for nearly 30\% of the recorded
individuals), then highest education level of the household held is
found quite significant for many commodities.


\subsubsection{Effect of Immigration}

With a significant migration from rural areas, one of the claims to
be tested is whether those resident in the community spend less on
positional consumption. While this does seem be to be a significant
factor, it has a weaker effect than age or household size (which is
to be further split as number of children and the number of members
minus the number of children) .


\subsubsection{Urbanization Effects}

Most of Tanzania appears to be sparsely populated with little access
to basic services and it is likely that the administrative classifications
of rural-urban areas do not reflect the consumer markets so well.
Still, ``is\_rural'' dummy is found significant for house-rent and
electricity (since most of rural Tanzania does not have electricity
- See Table \ref{tab:no_instruments_reg1}). 

If one were to use a dummy for accessible markets (created using the
distance from the surveyed household location to the closest daily
market ) - the effect of such a dummy is not so significant on positional
consumption. The region dummies - on the other hand - are found to
have more significance - indicating regional disparities for conspicuous
consumption in the country.


\subsubsection{Population density}

Population density is a crude measure for crowding in the cities.
The regions with higher population density do have a slight effect
on consumption of scarce commodities. It is hoped that a urban/rural
dummy created by classifying districts based on their population densities
(or at a finer granularity than regional levels) may give a more detailed
view on the effect of population density on conspicuous consumption.


\subsubsection{Services as Visible Consumption}

One of the interesting observations in the Vindex survey (Heffetz\cite{heffetz_visibility})
is the clustering of services and products. It is found that services
tend to be less ``visible'' in the Western consumer world. The clustering
might not be as clear-cut in the developing world - where social stratification
is severe and many services are contractual (non-monetary). The socio-cultural
barriers might have an effect through access to services.

Towards that claim, English education as a control parameter is found
quite significant for positional consumption. Those who identify themselves
as English speakers tend to spend more on scarce commodities. This
indicates that English education may be quite scarce - and while it
isn't reflected in the consumer expenditure market data so easily
- it's likely to play a role in status competitions.

\begin{figure}
\centering{}\includegraphics[scale=0.5]{lsms/lsms_visible_totexp_plot}\caption{Visible Expenditure vs Total Expenditure for LSMS 2010}
\end{figure}



\subsection{Analysis and Discussion}

Food is a significant portion of total spending overall \footnote{50\% of those surveyed spend 60\% or higher of their total expenditure
on food - subject to estimation errors.}. More importantly, those in non-agrarian professions spend about
as much of their total expenditure on food as those in agrarian occupations
\footnote{The median ratio of food-expenditure to total expenditure for agrarian
occupation households is 60\% while for non-agrarian occupations the
median is 57\%. Around 54\% of the total surveyed households were
in agrarian occupations.}. The other half of the expenditure is spent on housing, education
and energy requirements as well as various household products\footnote{Note that we may have slight errors in recording of food expenditure
due to extrapolation of the weekly diary}. 

While a commodity for private consumption (e.g. skin-cream or hobby-equipment
in the LSMS data) might have an appeal for everyone - whether it is
associated with high-income or not is a social psychological concern
and cannot be assessed from the household survey by itself. In the
absence of a visibility survey (asking the respondents how much they
notice a product and whether they associate the product with high-income
or not), one may still continue the discussion of the potential conspicuous
value of items by looking at how scarce the item is (based on the
percentile of consumers of the commodity). This is akin to repeating
the analysis of visible expenditure with a given commodity as the
only constituent of the visibility basket. The percentile of consumers
using a given commodity (e.g. top 22\% for electricity) and the slope
of $log(commodity-expenditure)$ vs $log(total-expenditure)$ can
tell us if richer sections of society spend more on a certain commodity
and if the poorer sections of society consume the chosen commodity
at all (the commodities chosen in the Table \ref{tab:regression_results}
are those where this slope is significant). The regression based on
data prepared from the last step attempts to calculate the coefficients
of the following equation:

\begin{equation}
ln(vis_{i})=\beta_{0}+\beta_{1}\cdot Dem_{i}+\beta_{2}\cdot ln(pInc_{i})+\epsilon\label{eq:lnvis_regression-2}
\end{equation}


Here $vis_{i}$ is the total visible consumption of the household
$i$ (expenditure on a chosen commodity such as electricity, sports
equipment), $Dem_{i}$ is a vector of demographic indicators under
consideration and $pInc_{i}$ is the permanent income - proxied by
total expenditure - which has been instrumented using $age$, $cubic(age)$,
$occupation$, $highest\_education$ level , $ln(highest\_education)$,
$cubic(highest\_education)$ \footnote{All 2sls regressions involved involved performing three diagnostic
tests provided by the function ivreg of package AER in R. These tests
are - i) a weak instrument test ii) a Wu-Hausman test for endogeneity
and iii) a Sargan test for validity of instruments.}.

Table \ref{tab:regression_results},\ref{tab:no_instruments_reg1}
and \ref{tab:instrumented_reg1}summarize the results obtained by
running regressions on several commodity-categories. A column in the
Table \ref{tab:regression_results} also suggests the percentile of
consumers using the commodity (electricity for example is used amongst
those having top 22\% of total expenditure). The usage of commodities
such as skincream and other-personal-products (shampoos, razors etc.)
are widespread compared with sports or hobby equipment and electricity.
For commodities that are rare and consumed only amongst the richer
sections of the society (those with higher total expenditure) the
effect of English literacy is significant. Similarly, hsize has a
significant effect on both educational expense and personal products
(using number of children instead of hsize could provide better association
with education expense).

We cannot claim from the results that the population spends more on
status commodities than education. What we can claim however, is that
electricity is more scarce than education. Further, in areas where
food is expensive, spending on marriage reduces - particularly by
the occupations that may bring higher incomes. This marks a preference
towards industrial goods in the urban (expensive) areas.

Another observation that can possibly help in modeling scarcity is
that scarcity of items seems to occur in clusters of objects. Carpets-rugs
require a certain housing status and access to English depends on
region. Similarly, many hobby equipments may require access to electricity
etc. The clustering of these items essentially point to the urban-rural
differences in the country.

\begin{table}
\centering{}%
\begin{tabular}{|>{\raggedright}p{4cm}||>{\raggedright}p{4cm}||>{\raggedright}p{2cm}||>{\raggedright}p{4cm}|}
\hline 
\multirow{1}{4cm}{Commodity} & Significant Variables & NonConsumer Percentile  & Variables significant after lnpinc instrumentation\tabularnewline
\hline 
\hline 
\multirow{1}{4cm}{carpetsrugs} & lnpinc, age, hsize, housingstatus, highest\_educ, english & 78 & lnpinc, age,hsize,

highest\_educ,

english\tabularnewline
\hline 
\hline 
educexpense & lnpinc, age, hsize, housingstatus, occupation & 35 & lnpinc,age,

hsize,

housingstatus,

occupation\tabularnewline
\hline 
\hline 
electricity & lnpinc, age, hsize, housingstatus, occupation, isrural, highest\_educ,
region, english, is\_resident & 78 & Chosen instruments (occupation, ln\_highest\_educ ) 

did not demonstrate endogeneity of lnpinc\tabularnewline
\hline 
\hline 
houserent & lnpinc, age, housingstatus, roomsnum & 84 & lnpinc,

housingstatus\tabularnewline
\hline 
\hline 
personal items repair & lnpinc, highest\_educ, region & 96 & lnpinc, highest\_educ, region\tabularnewline
\hline 
\hline 
personal products & lnpinc, hsize, roomsnum, years\_community & 37 & lnpinc, hsize, roomsnum, years\_community\tabularnewline
\hline 
\hline 
skin cream & lnpinc, age, hsize, isrural, region, years\_community & 12 & lnpinc, age, hsize, region, years\_community\tabularnewline
\hline 
\hline 
funeral costs & lnpinc, region, roomsnum & 54 & lnpinc, region, roomsnum\tabularnewline
\hline 
\hline 
marriage costs & lnpinc, region, english, roomsnum, years\_community & 75 & lnpinc, region, english, roomsnum, years\_community\tabularnewline
\hline 
\hline 
sports and hobby equipment & lnpinc, age, housingstatus, region, english & 93 & lnpinc, age, housingstatus, region, english\tabularnewline
\hline 
\end{tabular}\caption{\label{tab:regression_results}Results from regression over selected
variables}
\end{table}

\begin{landscape}
\begin{table}
[!htbp] \centering 
\caption{\label{tab:no_instruments_reg1} Regression for scarce commodities with no instrumentation}
\resizebox{\columnwidth}{!}{
\begin{tabular}{@{\extracolsep{5pt}}lcccccccccc} 
\\[-1.8ex]\hline 
\hline \\[-1.8ex] 
 & \multicolumn{10}{c}{\textit{Dependent variable: consumption}} \\ 
\cline{2-11} 
\\[-1.8ex] & \multicolumn{10}{c}{depvar} \\ 
\\[-1.8ex] & carpetsrugs(1) & education(2) & electricity(3) & houserent(4) & personalitemsrepair(5) & personalprods(6) & skincream(7) & funeral(8) & marriage(9) & hobbyequipment(10)\\ 
\hline \\[-1.8ex] 
 lnpinc & 4.708$^{***}$ & 3.574$^{***}$ & 4.391$^{***}$ & 1.154$^{***}$ & 0.843$^{***}$ & 3.439$^{***}$ & 2.145$^{***}$ & 2.759$^{***}$ & 3.296$^{***}$ & 1.214$^{***}$ \\ 
  & (0.328) & (0.239) & (0.332) & (0.173) & (0.170) & (0.281) & (0.207) & (0.260) & (0.261) & (0.142) \\ 
  & & & & & & & & & & \\ 
 age & $-$0.106$^{***}$ & 0.086$^{***}$ & 0.067$^{***}$ & $-$0.067$^{***}$ &  &  & $-$0.042$^{***}$ &  &  & $-$0.038$^{***}$ \\ 
  & (0.023) & (0.017) & (0.020) & (0.011) &  &  & (0.015) &  &  & (0.010) \\ 
  & & & & & & & & & & \\ 
 hsize & $-$0.459$^{***}$ & 2.160$^{***}$ & $-$0.529$^{***}$ &  &  & $-$0.506$^{***}$ & 0.217$^{***}$ &  &  &  \\ 
  & (0.115) & (0.089) & (0.102) &  &  & (0.104) & (0.067) &  &  &  \\ 
  & & & & & & & & & & \\ 
 housingstatus & 0.600$^{***}$ & $-$1.049$^{***}$ & 0.924$^{***}$ & 4.280$^{***}$ &  &  &  &  &  & 0.452$^{***}$ \\ 
  & (0.208) & (0.187) & (0.191) & (0.131) &  &  &  &  &  & (0.106) \\ 
  & & & & & & & & & & \\ 
 occupation\_rank &  &  & 0.782$^{***}$ &  &  &  &  &  &  &  \\ 
  &  &  & (0.295) &  &  &  &  &  &  &  \\ 
  & & & & & & & & & & \\ 
 isrural &  &  & $-$6.468$^{***}$ & $-$3.501$^{***}$ &  &  & 1.469$^{***}$ &  &  &  \\ 
  &  &  & (0.642) & (0.419) &  &  & (0.465) &  &  &  \\ 
  & & & & & & & & & & \\ 
 highest\_educ & $-$0.295$^{***}$ &  & 0.421$^{***}$ &  & 0.075$^{**}$ &  &  &  &  &  \\ 
  & (0.076) &  & (0.066) &  & (0.035) &  &  &  &  &  \\ 
  & & & & & & & & & & \\ 
 region &  &  & 0.186$^{***}$ & $-$0.051$^{***}$ & $-$0.049$^{***}$ &  & $-$0.121$^{***}$ & $-$0.142$^{***}$ & $-$0.034$^{**}$ & $-$0.066$^{***}$ \\ 
  &  &  & (0.017) & (0.011) & (0.010) &  & (0.012) & (0.018) & (0.016) & (0.009) \\ 
  & & & & & & & & & & \\ 
 english & 3.146$^{***}$ &  & 2.949$^{***}$ &  &  &  &  &  & 1.976$^{**}$ & 1.633$^{***}$ \\ 
  & (0.953) &  & (0.840) &  &  &  &  &  & (0.794) & (0.455) \\ 
  & & & & & & & & & & \\ 
 roomsnum &  &  &  & $-$0.919$^{***}$ &  & 0.442$^{***}$ &  & 0.625$^{***}$ & 0.654$^{***}$ &  \\ 
  &  &  &  & (0.100) &  & (0.169) &  & (0.157) & (0.146) &  \\ 
  & & & & & & & & & & \\ 
 is\_resident &  &  & $-$1.956$^{***}$ & $-$1.977$^{***}$ &  &  &  &  &  &  \\ 
  &  &  & (0.558) & (0.366) &  &  &  &  &  &  \\ 
  & & & & & & & & & & \\ 
 years\_community &  &  &  &  &  & $-$0.073$^{***}$ & $-$0.026$^{**}$ &  & $-$0.054$^{***}$ &  \\ 
  &  &  &  &  &  & (0.015) & (0.013) &  & (0.014) &  \\ 
  & & & & & & & & & & \\ 
 Constant & $-$71.251$^{***}$ & $-$64.314$^{***}$ & $-$85.424$^{***}$ & $-$31.269$^{***}$ & $-$33.945$^{***}$ & $-$47.620$^{***}$ & $-$21.851$^{***}$ & $-$46.797$^{***}$ & $-$61.169$^{***}$ & $-$36.167$^{***}$ \\ 
  & (4.287) & (3.438) & (4.610) & (2.689) & (2.214) & (4.026) & (3.020) & (3.654) & (3.767) & (2.098) \\ 
  & & & & & & & & & & \\ 
\hline \\[-1.8ex] 
Observations & 2,240 & 2,965 & 2,240 & 2,965 & 2,240 & 2,965 & 2,965 & 2,965 & 2,963 & 2,963 \\ 
R$^{2}$ & 0.126 & 0.322 & 0.437 & 0.502 & 0.029 & 0.084 & 0.094 & 0.059 & 0.101 & 0.073 \\ 
Adjusted R$^{2}$ & 0.124 & 0.321 & 0.435 & 0.501 & 0.027 & 0.082 & 0.092 & 0.058 & 0.100 & 0.071 \\ 
Residual Std. Error & 13.394 (df = 2233) & 13.463 (df = 2960) & 11.595 (df = 2229) & 8.929 (df = 2957) & 7.386 (df = 2236) & 14.919 (df = 2960) & 10.078 (df = 2958) & 15.518 (df = 2961) & 13.840 (df = 2957) & 7.963 (df = 2957) \\ 
F Statistic & 53.824$^{***}$ (df = 6; 2233) & 351.136$^{***}$ (df = 4; 2960) & 173.281$^{***}$ (df = 10; 2229) & 426.503$^{***}$ (df = 7; 2957) & 21.953$^{***}$ (df = 3; 2236) & 67.429$^{***}$ (df = 4; 2960) & 51.003$^{***}$ (df = 6; 2958) & 61.522$^{***}$ (df = 3; 2961) & 66.629$^{***}$ (df = 5; 2957) & 46.343$^{***}$ (df = 5; 2957) \\ 
\hline 
\hline \\[-1.8ex] 
\textit{Note:}  & \multicolumn{10}{r}{$^{*}$p$<$0.1; $^{**}$p$<$0.05; $^{***}$p$<$0.01} \\ 
\end{tabular} 
}
\end{table}
\end{landscape}


\begin{landscape}
\begin{table}
[!htbp] \centering
\caption{\label{tab:instrumented_reg1}Instrumented Regression for scarce commodities}
\resizebox{\columnwidth}{!}{
\begin{tabular}{@{\extracolsep{5pt}}lcccccccccc} 
\\[-1.8ex]\hline 
\hline \\[-1.8ex] 
 & \multicolumn{10}{c}{\textit{Dependent variable:}} \\ 
\cline{2-11} 
\\[-1.8ex] & lnvis & lndseducexpense & lnvis & lndshouserent & \multicolumn{6}{c}{lnvis} \\ 
\\[-1.8ex] & carpetsrugs(1) & education(2) & electricity(3) & houserent(4) & personalitemsrepair(5) & personalprods(6) & skincream(7) & funeral(8) & marriage(9) & hobbyequipment(10)\\ 
\hline \\[-1.8ex] 
 lnpinc & 4.665$^{***}$ & 3.033$^{***}$ & 9.941$^{***}$ & 0.982$^{**}$ & 0.747$^{**}$ & 3.216$^{***}$ & 1.661$^{***}$ & 2.770$^{***}$ & 3.446$^{***}$ & 1.593$^{***}$ \\ 
  & (0.657) & (0.597) & (1.247) & (0.432) & (0.321) & (0.565) & (0.502) & (0.484) & (0.627) & (0.318) \\ 
  & & & & & & & & & & \\ 
 age & $-$0.106$^{***}$ & 0.081$^{***}$ & 0.055$^{***}$ & $-$0.074$^{***}$ &  &  & $-$0.040$^{**}$ &  &  & $-$0.060$^{***}$ \\ 
  & (0.023) & (0.017) & (0.021) & (0.016) &  &  & (0.020) &  &  & (0.014) \\ 
  & & & & & & & & & & \\ 
 hsize & $-$0.454$^{***}$ & 2.227$^{***}$ & $-$1.182$^{***}$ &  &  & $-$0.518$^{***}$ & 0.346$^{***}$ &  &  &  \\ 
  & (0.131) & (0.112) & (0.178) &  &  & (0.140) & (0.099) &  &  &  \\ 
  & & & & & & & & & & \\ 
 housingstatus & 0.605$^{***}$ & $-$0.979$^{***}$ & 1.028$^{***}$ & 4.402$^{***}$ &  &  &  &  &  & 0.491$^{***}$ \\ 
  & (0.217) & (0.200) & (0.203) & (0.157) &  &  &  &  &  & (0.127) \\ 
  & & & & & & & & & & \\ 
 occupation\_rank &  &  & $-$0.723 &  &  &  &  &  &  &  \\ 
  &  &  & (0.451) &  &  &  &  &  &  &  \\ 
  & & & & & & & & & & \\ 
 isrural &  &  & $-$3.861$^{***}$ & $-$3.618$^{***}$ &  &  & 0.951 &  &  &  \\ 
  &  &  & (0.883) & (0.585) &  &  & (0.626) &  &  &  \\ 
  & & & & & & & & & & \\ 
 highest\_educ & $-$0.292$^{***}$ &  & 0.132 &  & 0.084$^{*}$ &  &  &  &  &  \\ 
  & (0.089) &  & (0.094) &  & (0.044) &  &  &  &  &  \\ 
  & & & & & & & & & & \\ 
 region &  &  & 0.187$^{***}$ & $-$0.057$^{***}$ & $-$0.049$^{***}$ &  & $-$0.106$^{***}$ & $-$0.138$^{***}$ & $-$0.054$^{***}$ & $-$0.076$^{***}$ \\ 
  &  &  & (0.018) & (0.014) & (0.010) &  & (0.014) & (0.021) & (0.020) & (0.012) \\ 
  & & & & & & & & & & \\ 
 english & 3.155$^{***}$ &  & 2.263$^{**}$ &  &  &  &  &  & 2.253$^{**}$ & 1.574$^{***}$ \\ 
  & (0.962) &  & (0.903) &  &  &  &  &  & (0.984) & (0.577) \\ 
  & & & & & & & & & & \\ 
 roomsnum &  &  &  & $-$1.020$^{***}$ &  & 0.589$^{***}$ &  & 0.412$^{**}$ & 0.518$^{***}$ &  \\ 
  &  &  &  & (0.132) &  & (0.197) &  & (0.186) & (0.183) &  \\ 
  & & & & & & & & & & \\ 
 is\_resident &  &  & $-$0.369 & $-$2.191$^{***}$ &  &  &  &  &  &  \\ 
  &  &  & (0.684) & (0.494) &  &  &  &  &  &  \\ 
  & & & & & & & & & & \\ 
 years\_community &  &  &  &  &  & $-$0.077$^{***}$ & $-$0.033$^{*}$ &  & $-$0.065$^{***}$ &  \\ 
  &  &  &  &  &  & (0.020) & (0.017) &  & (0.020) &  \\ 
  & & & & & & & & & & \\ 
 Constant & $-$70.746$^{***}$ & $-$56.921$^{***}$ & $-$156.675$^{***}$ & $-$28.337$^{***}$ & $-$32.733$^{***}$ & $-$44.482$^{***}$ & $-$15.511$^{**}$ & $-$46.113$^{***}$ & $-$62.420$^{***}$ & $-$40.656$^{***}$ \\ 
  & (7.971) & (8.229) & (16.113) & (6.362) & (4.088) & (8.068) & (7.060) & (6.874) & (9.036) & (4.486) \\ 
  & & & & & & & & & & \\ 
\hline \\[-1.8ex] 
Observations & 2,240 & 2,965 & 2,240 & 2,240 & 2,240 & 2,240 & 2,240 & 2,240 & 2,240 & 2,240 \\ 
R$^{2}$ & 0.126 & 0.321 & 0.367 & 0.502 & 0.028 & 0.069 & 0.080 & 0.043 & 0.090 & 0.078 \\ 
Adjusted R$^{2}$ & 0.124 & 0.320 & 0.364 & 0.500 & 0.027 & 0.067 & 0.077 & 0.042 & 0.088 & 0.076 \\ 
Residual Std. Error & 13.394 (df = 2233) & 13.474 (df = 2960) & 12.299 (df = 2229) & 9.500 (df = 2232) & 7.386 (df = 2236) & 14.768 (df = 2235) & 9.766 (df = 2233) & 15.740 (df = 2236) & 14.286 (df = 2234) & 8.484 (df = 2234) \\ 
\hline 
\hline \\[-1.8ex] 
\textit{Note:}  & \multicolumn{10}{r}{$^{*}$p$<$0.1; $^{**}$p$<$0.05; $^{***}$p$<$0.01} \\ 
\end{tabular}
}
\end{table}
\end{landscape}


\part{A Behavioural Model for Status Utility}


\section{A model for Utility and Status}

The concept of status is rather non-trivial and has characteristics
of a feedback system in the long-run (status may yield income through
social barriers but requires income to be acquired). The Ireland model
used in the literature treats status-signaling as purchasing of visible
and non-visible goods\cite{ireland_signals}. In the Ireland model,
the combined utility for every consumer is $U=F(f(v,w),s)$ where
$f(v,w)$ is the private utility of the consumer and status $s$ is
assumed to be an increasing function of inference of others - $s=f(v,g(v)$)
- with $v$ denoting visual consumption and $w$ - the consumption
that is not directly observable. Every consumer thus optimizes the
combined private and visible utility. A practical consideration in
the model is the separation between visible and non-visible consumption
- a boundary that requires a socio-culutral judgment and has been
drawn using consumer surveys in the literature.

What research in the developing markets further points out is that
the parameter of combined utility in a simplified model - ($U=(1-a)\cdot f(v,w)+a\cdot f(v,g(v)$)
- can vary for different sections of society. A slight adjustment
of the model may be to add another parameter that indicates the consumer's
social class. This may be relevant in the developing world where social
status is yielded through social barriers. Sections of society that
are endowed with a higher status capital may be in less of a need
to purchase commodities of visible consumption.

In the analysis of LSMS data on Tanzania so far, urban/rural differences
have significance in the consumption of scarce commodities. There
are two ways to incorporate this into a model of signaling - one is
to consider these household characteristics as a social class of consumers
so that different visible sensitivities for social classes is noted.
The other is to consider these characteristics as part of conspicious
consumption in the long-run. For example, English literacy seems to
have correlation with the consumption of certain scarce products in
Tanzania. In the model of scarcity, English literacy (along with urban
residence and other characteristics significant for consumption of
scarce commodities) would be seen as a status commodity (capital)
that is acquired through spending on education or migration (or other
relevant commodities).


\subsection{A Word of Caution}

Notice that one needs to be careful while drawing conclusions based
on consumption of commodities that are themselves selected based on
the percentiles of consumption levels. The threshold method that we
use to select scarce commodities (that are likely to be visible) -
considers items that i) are accessible by no more than 50\% of the
consumers and ii) have their expenditure rising with permanent income.
These are - by definition - items that the rich are more likely to
afford. We cannot select items that the only richer section of society
indulges in and claim that people spending on these selected items
indicate higher status. Such a claim is only a restatement of the
high permanent income and says nothing more substantial than that
the richer population sections signal higher status. It would be a
fallacy to associate visible consumption with household characteristics
by only associating household characteristics with permanent income.
The threshold method that we use only measures the ``scarcity''
of the item (e.g. electricity is more scarce than food) - not status
or visibility per se - which involve some socio-cultural judgement.
That scarcity itself has an indirect effect on status competitions
cannot be denied - but this effect is not measured by the threshold
method of classifying alone.


\section{A model for scarcity and congestion}

In recent decades, the urban settlements in Africa have seen massive
overpopulation and development of the services sector. The differences
in urban-rural lifestyles have increased. The scarcity of services
and of industrial goods (which would be a necessity in the Western
world) seem evidently sparse in the developing countries (a claim
that is verified by data on Tanzania).

A relevant question amidst these developments is whether a consumer
prefers a larger house over installation of electricity or not. As
a commodity, electricity is both scarce and visible - as it opens
up more lifestyle choices. A survey detailed in the next sections
aims to test the presence of a preference for electricty (and other
scarce goods) - but as is, the data suggests significant urban-rural
differences across regions in Tanzania. In a Hirschian sense, a congestion
\cite{HirschSocialLimits} is likely to exist for electricity. 

A general view on scarcity would allow us to classify the commodities
based on their scarcity and quantify the urban-rural differences better.
The survey detailed in next sections may further help measure the
impact of such scarcities on consumer preferences. It is noted that
many of the items are scarce together (carpets and housing-status,
electricty and hobby equipments, etc.) . It is through such denial
of goods and services that status perceptions develop. A simple view
of observed scarcities can be provided by a directed graph of items
for classification - where a node is an item and points to other nodes/items
that it denies (which themselves can be formed with items that deny
other items and so and so forth). For example, electricity can be
a node in this graph with connections to equipments and electronics
but no connection to food items. The disconnected nodes in the graph
- would be least likely to be affected by another unreachable node
in the graph. The criteria of connections (denials) would be determined
by statistical significance e.g. rice and walnuts appear would not
be scarce together if one of them is available and affordable by both
higher and lower quantiles of permanent income of society.


\section{A Behavioural Experiment for Status Competitions}


\subsection{Status and Consumption as games}

Behavioural games have been used in the developing countries to gauge
motivations of the participating consumers \footnote{A study by Sophie Clot studies the effect high and low effort work
on consumption by conducting an experiment at the payment office where
some amount of pay is distributed for low-effort work and some for
high-effort work.}. While the visibility surveys (\cite{ZahraIndia,heffetz_visibility})
attempt to study how consumption on certain commodities may signal
status, the goal of the proposed game is to characterise environments
under which the perceptions of a higher-status may develop. The game
attempts to emulate i) the consumer market and ii) the mechanism through
which status may be assigned within a group of consumers. It therefore
relies on participants playing the dual role of a consumer and status-observer.

The activities of purchasing and assigning status are separate in
the game. Since a simulated purchase performed by the participants
in the game (given a list of commodities, prices and outlay) is quite
likely to deviate from their real world purchases and their real needs,
the participants are instead asked what additional items they would
purchase for a given a basket of commodities that they already possess
(using a cumulative voting scheme that emulates selection of commodities
in a market - see section \ref{sub:Purchasing_Mechanism} for details).
The second part of the game emulates status assignment - where participants
assign a score of status and effectiveness each to 3 (or more) other
participants in the game by looking at the quantities of the item
categories possessed and purchased by the latter. The judgment of
status in the real world does not involve direct observance of prices
and thus it is only the quantity of the identified items consumed
or already possessed that matters in the status-assignment part of
the game. The end-goal of the game is to purchase a basket most desired
by others - the winner achieves this goal by purchasing commodities
of her choice that are most desirable by everyone and are indicative
of a rank higher than everyone else in the game.


\subsection{\label{sub:Purchasing_Mechanism}Purchasing Mechanism}

It is difficult for players to conduct a ``simulated shopping''
in a way that truly represents their needs. Hence, instead of asking
the respondent how they'll spend the given outlay of a 1000 dollars
over a set of commodities, they are asked how they would spend the
additional 100 dollars for a given 1000 dollars of outlay (or more)
value of items that they already have stocked up. The ``stock''
items can be chosen by the players as a first step in the game and
is intended to match their own consumption pattern. While the ``stock''
is made of non-positional items, the participants choose 3 items from
a mix of non-positional items and positional items - given the 10\%
extra outlay. Since all participants cannot be assumed to be equally
numerate, the game uses a scheme similar to cumulative voting - where
10 virtual coins are provided to the participant and the participant
is asked to distribute the coins amongst a set of available items
(both positional and non-positional). The provided outlay in the game
(number of coins) may vary for participants - in proportion to the
income distribution that is observed in the relevant consumption surveys
(e.g. LSMS for Tanzania).

In summary the following steps are taken in the game:
\begin{enumerate}
\item \label{enu:step_stock}Choose a stock basket that is closest to one's
own consumption pattern (no more than 5 basket classifications are
provided to choose from)
\item Acknowledge the real-life constraint (see Section \ref{sub:constraints})
\item Use the given additional outlay (10 or less virtual coins) to purchase
and add (positional and well as non-positional items) to the strictly
non-positional stock basket that was selected in the step \ref{enu:step_stock}\footnote{It is necessary to estimate the price of products and services for
the purchasing game to emulate the market.}
\item Provide a score (1..5) on effectiveness and status to 3 other participants
whose total outlay and the choice of items purchased (along with number
of coins used for every item) is also known
\end{enumerate}

\subsubsection{\label{sub:basket_mixes}Mixes in the Consumer Basket}

While the basket for every consumer can be varied to model urban/rural
differences or the distance /accessibility of the particular commodity
classes, the game ensures that all participants have reasonably similar
consumer universe. Consequently, no category is intended to be completely
removed from the basket (i.e. all baskets have the same set of categories).
Following are the categories for which the positional/non-positional
variants are sought:
\begin{enumerate}
\item Food - Fruits, Meat, Baked Goods or Nuts/Cereals and Pulses, Milk
(minor items such as salt and spices are not included), Tea, Soda/
Beer and Wine
\item Household products (Detergent, Electronics)
\item Personal Products (Clothes, Shoes, Makeup)
\item Household services (House refurbishments) and Energy (electricity/kerosene)
\item Savings for future Asset purchase
\item Entertainment/Dining Out/Travel/Travel Abroad
\item Health
\item Education (School/University)
\end{enumerate}

\subsubsection{\label{sub:constraints}Constraints and Assets}

The game attempts to measure status and consumption with respect to
high asset ownership, social class or familial responsibility. Since
players choose between physical needs and positional needs in the
game, a different circumstance is likely to affect their choice and
hence their perceived status. The game presents a precondition to
the player - indicating high asset ownership, a chosen social class
or a familial liability. For example, to test a participant's choice
between food and electricity, the game can present a large family
as a constraint, and record the choice between spending more on food
vs installing electricity. The game thus measures indirect effects
of reward or constraints on status by allowing participants to gauge
the suitability of a participant's choice in the status game in the
presence of constraints (familial) or rewards (asset-related).

Notice that the constraint variable is only planned to be binary in
the current scheme i.e. it is either a reward or a liability (when
present). The two values are expected to have an opposite affect on
the purchase of new items. Admittedly, the binary values of constraints
vs rewards circumvent the difficulty in comparisons between disparate
needs of the consumers - e.g. a large family, senior member or a social
event (e.g. marriage/funeral). While a multivalued variable (if adopted)
can potentially provide better insights into the relative effects
of these several types constraints, the goal of the current exercise
is to test for a direct effect of constraints on status (rather than
relative effect of the various possible constraints).


\subsection{Status ranking}

The status-ranking activity involves a student assigning a status
score by looking at i) what the other participant with a given income
level does with the extra outlay and ii) what the participant already
possesses. In the ranking scheme, the participants provide a score
on effectiveness as well as status to all the other (3 or more) participants
observed. Notice that in presence of constraints specified in section
\ref{sub:constraints}, regardless of whether one is selfish or not,
a participant would tend to penalise someone else who she thinks is
going to be more selfish than herself. Since the game provides a way
to penalize selfishness by status ranks, the participants are discouraged
from indicating status through overspending on positional items. The
penalty for not caring for a sick parent may be huge in the society
but so can be the penalty for being stingy. Similarly, while some
may want to indicate wealth by buying a watch they may also fear disrespect
for not taking care of a sick family member. The scores on effectiveness
and status are thus not only a way to discourage the consumer from
limiting the unrealistic purchases in the simulated purchasing part
of the game, they also track the effect of the externalities such
as sickness or age (measured through the binary variable discussed
in section \ref{sub:constraints}).

While consumers try to maximise their utility by purchasing more items
for a given limited outlay - they also manage their prestige by letting
others have a better opinion of themselves. The status game can thus
be seen as an enhanced version of the survey that asks people to imagine
a neighbour who spends more than them on a chosen commodity (used
in \cite{ZahraIndia,heffetz_visibility}). The proposed game attempts
to measure how consumers might act given a certain circumstances while
both status and welfare (effectiveness score can be seen as a proxy
of concern for others) become part of the payoff function in the game. 


\subsection{Welfare and Status competitions}

The solution of this game for a set of rational players remains a
pending exercise in this study. The key motivation for the analysis
at this point is that fundamentally all social welfare concerns are
concerns of Pareto optimality. Moreover, the payoff function for effectiveness
in the game is meant to be a proxy for welfare. 

With Pareto optimality in mind, more spending on education, health
seems desirable - but it may be become distant for consumers due to
their immediate needs - whether positional or non-positional. A comparison
with what is observed in consumption data versus what is observed
about positional consumption in games can provide some insight into
the social status that can influence the desired welfare equilibrium.


\subsection{Survey Questionnaire}

You have 10,000 (or 100) to spend today. What are the objects that
you would purchase if you were to enter the market today? Please take
a look at the constraints that might affect your consumption. Try
choosing the smartest way possible - the prices. You would also need
to compare 2 other candidates as part of this game (as others would
rate you). Try being close to your real circumstances. Unrealistic
values may disqualify you from the game.


\section{\label{sec:status_consumption_need}Policy implications}

The discussion so far leans towards permitting status competitions
rather than attempting to tax or control them. This is in line with
the suggestions offered by Robert H Frank \cite{FrankPond} favouring
a non-monetary market of stauses only so that status games (which
are a necessity) do not overlap with the market for real goods. Due
to structural reasons of the modern economy, advertising efforts can
turn a social scarcity into a physical scarcity (to use the Hirsch's
terminology\cite{HirschSocialLimits}). A profit-driven industry and
the advertising pursued by the companies tend to increase the status
competition for a commodity. Instead of letting status competitions
modify the distribution of that physical goods through competition
(and thus do little to avoid the problem of physical scrarcities in
the developing countries), policy can attempt to provide status-games
in a world of non-necessity items - in some ways to diffuse the status
competitions in the society.

In poor and non-pecuniary societies, the desire to become rich or
the benefit of inheriting money and education is often less reachable.
Status and money translate into social securities in unstructured
societies. These may well be detected in the countries in Africa -
but limited data on household characteristics in Tanzania (related
to ethnicity or religion) have prevented us from such an analysis
for Tanzania.

The question that we seek the answer for in the context of Tanzania
(or another developing country) is whether the expenditure on high-status
or scarce items (an analysis similar to one conducted by Prais Houthakker
for expensive and cheap tea varieties amongst social classes in the
UK\cite{HouthakkerFamilySurvey}) - is actually more desirable than
on housing and education. The designed experiment intends to find
answer to this question. If the answer is indeed the former, then
it makes sense to limit the status competitions through policy to
support status competitions on non-essential items (possibly by introducing
brand differentiation). Attaching glamour to education, healthcare
and food items may help consumers prioritize their needs.


\part{Effect of Price on food vs non-food items}


\section{Price Changes}

The literature has not used panel data analysis in the context of
conspicuous consumption. While an influence of rising prices can complicate
the analysis of visible consumption indicators, the insights from
demand elasticities are essential to understanding the relative effect
of status-related consumption against other commodities. Higher price
of food items may suppress consumption on food - but one cannot answer
whether an increase in price of food suppresses its consumption more
than it suppresses consumption of non-food items or not - without
an estimation of demand elasticities. Such details of consumption
patterns are basket-dependent and are not accessible without a record
of prices of all types of items in the basket. Unfortunately, a lack
of prices for non-food items in the LSMS prevent this much desired
time-series analysis.

Even though an analysis on non-food prices is inaccessbile with the
unavailability of price data in most consumption surveys (e.g. LSMS),
a time-series analysis based on food prices alone can provide insights
into the pressures on food consumption. Using historical prices on
calorie consumption in India, Deaton and Jean Dreze point out that
the overall calorie consumption has declined while the total outlay
has increased in India (\cite{DeatonDrezeIndiaFood}). The change
in positional value of food - determined by price differentiation
in the market and scarcity - can potentially help explain some of
this decline. While such a decline is reported to be less in the case
of sub-Saharan African countries than in India, the regional differences
within the country could be explained by the change of food's position
in the consumer universe (i.e. the so-called ``Sen argument''\cite{DeatonDrezeIndiaFood}).

Congestion - a related phenomenon - is subject to demand and supply
for a particular commodity and can be measured. If we were to consider
food, for example, a limited supply and overpopulation can increase
competition. Similarly, for entertainment, censorship and introduction
of internet can create new competitions (congestion). For housing,
new constructions and overpopulation can cause congestion. These are
commodity-specific instances and a focus on selected items may be
the only way to test whether the changes in consumption patterns for
the chosen commodity are explained by new scarce items and the competition
caused for them. The data from Tanzania - so far - only seems to point
that availability of services in urban and rural area can potentially
cause some congestion (competitions for scarce items).


\section{Food prices from LSMS - a preliminary analysis}

Scarcity is interpreted in terms of availability and affordability
in the study. The geographical regions may need to be understood in
terms of scarcity. Further, population density and migration data
may provide better insights in the interplay of food and non-food
consumption.

It is noted that in certain areas in Tanzania - prices for food vary
a lot more than they do in others. This is a phenomenon that varies
from commodity to commodity. For example, the prices for onions and
sugar don't vary so much by area code as they do for meat and chicken.
The regions Dar-es-salaam, Mbeya mwanza, Mjini/Magharini unguja stand
out for higher prices for multiple items. In a preliminary analysis,
a indicator dummy for these regions is found significant - but it
is also noted that these areas are urban settlements where electricity
is available and population is significantly high (See Table \ref{tab:popreg-reg2-simple2}).

Certain food items for example, have more price-differences overall
than others - rice (husked), maize(grain), sweet potatoes, Irish potatoes,
groundnuts(shelled), goat meat, chicken and canned milk correspond
to numerous (>4) region-codes where they're reportedly sold in different
prices ranges. While it is tempting to claim that price differences
in the market indicate that there is more price-differentiation and
possibly more competition - one needs to consider the overall scarcity
of the commodity (the percentiles of the commodity expenditure in
the threshold method) as well as the preference for the item amongst
the rich (measured by higher expenditure with income) for the item
to be considered a status-signaling item. 


\begin{landscape}
\begin{table}
[!htbp] \centering 
\caption{\label{tab:popreg-reg2-simple2}No instruments regression with population density and expensive-food dummy included}
\resizebox{\columnwidth}{!}{
\begin{tabular}{@{\extracolsep{5pt}}lcccccccccc} 
\\[-1.8ex]\hline 
\hline \\[-1.8ex] 
 & \multicolumn{10}{c}{\textit{Dependent variable:}} \\ 
\cline{2-11} 
\\[-1.8ex] & \multicolumn{10}{c}{depvar} \\ 
\\[-1.8ex] & carpetsrugs(1) & education(2) & electricity(3) & houserent(4) & personalitemsrepair(5) & personalprods(6) & skincream(7) & funeral(8) & marriage(9) & hobbyequipment(10)\\ 
\hline \\[-1.8ex] 
 lnpinc & 5.095$^{***}$ & 3.979$^{***}$ & 3.940$^{***}$ & 0.991$^{***}$ & 0.667$^{***}$ & 3.439$^{***}$ & 1.894$^{***}$ & 1.644$^{***}$ & 2.494$^{***}$ & 1.017$^{***}$ \\ 
  & (0.365) & (0.269) & (0.342) & (0.171) & (0.128) & (0.281) & (0.253) & (0.322) & (0.275) & (0.159) \\ 
  & & & & & & & & & & \\ 
 age & $-$0.114$^{***}$ & 0.086$^{***}$ &  & $-$0.070$^{***}$ &  &  & $-$0.042$^{**}$ &  & $-$0.036$^{**}$ & $-$0.043$^{***}$ \\ 
  & (0.023) & (0.017) &  & (0.011) &  &  & (0.019) &  & (0.017) & (0.010) \\ 
  & & & & & & & & & & \\ 
 hsize & $-$0.522$^{***}$ & 2.109$^{***}$ & $-$0.404$^{***}$ &  &  & $-$0.506$^{***}$ & 0.285$^{***}$ &  &  &  \\ 
  & (0.116) & (0.090) & (0.104) &  &  & (0.104) & (0.082) &  &  &  \\ 
  & & & & & & & & & & \\ 
 housingstatus & 0.675$^{***}$ & $-$0.923$^{***}$ & 0.871$^{***}$ & 4.250$^{***}$ & 0.191$^{**}$ &  &  &  &  & 0.413$^{***}$ \\ 
  & (0.213) & (0.190) & (0.192) & (0.131) & (0.091) &  &  &  &  & (0.111) \\ 
  & & & & & & & & & & \\ 
 occupation\_rank &  &  & 1.012$^{***}$ &  &  &  &  & $-$0.681$^{**}$ &  &  \\ 
  &  &  & (0.296) &  &  &  &  & (0.320) &  &  \\ 
  & & & & & & & & & & \\ 
 isrural &  &  & $-$3.120$^{***}$ & $-$3.952$^{***}$ &  &  &  &  &  &  \\ 
  &  &  & (0.660) & (0.410) &  &  &  &  &  &  \\ 
  & & & & & & & & & & \\ 
 highest\_educ & $-$0.284$^{***}$ &  & 0.472$^{***}$ &  &  &  & $-$0.120$^{**}$ &  &  &  \\ 
  & (0.075) &  & (0.067) &  &  &  & (0.047) &  &  &  \\ 
  & & & & & & & & & & \\ 
 expensiveregion &  &  & 3.354$^{***}$ &  &  &  &  &  & $-$1.517$^{**}$ & $-$1.331$^{***}$ \\ 
  &  &  & (0.751) &  &  &  &  &  & (0.764) & (0.449) \\ 
  & & & & & & & & & & \\ 
 popdensity & $-$0.001$^{***}$ & $-$0.001$^{***}$ & 0.001$^{***}$ &  & 0.0002$^{*}$ &  &  & 0.003$^{***}$ & 0.003$^{***}$ & 0.001$^{***}$ \\ 
  & (0.0003) & (0.0002) & (0.0003) &  & (0.0001) &  &  & (0.0003) & (0.0003) & (0.0002) \\ 
  & & & & & & & & & & \\ 
 english & 2.822$^{***}$ &  & 4.425$^{***}$ &  &  &  &  &  & 1.913$^{**}$ & 1.132$^{**}$ \\ 
  & (0.951) &  & (0.840) &  &  &  &  &  & (0.774) & (0.453) \\ 
  & & & & & & & & & & \\ 
 years\_community &  &  & 0.094$^{***}$ &  &  & $-$0.073$^{***}$ & $-$0.036$^{**}$ &  &  &  \\ 
  &  &  & (0.020) &  &  & (0.015) & (0.015) &  &  &  \\ 
  & & & & & & & & & & \\ 
 roomsnum &  &  &  & $-$0.915$^{***}$ &  & 0.442$^{***}$ &  & 0.923$^{***}$ & 0.936$^{***}$ &  \\ 
  &  &  &  & (0.101) &  & (0.169) &  & (0.166) & (0.150) &  \\ 
  & & & & & & & & & & \\ 
 is\_resident & $-$1.873$^{***}$ &  & $-$2.607$^{***}$ & $-$2.113$^{***}$ &  &  &  &  &  & $-$0.683$^{**}$ \\ 
  & (0.650) &  & (0.732) & (0.366) &  &  &  &  &  & (0.335) \\ 
  & & & & & & & & & & \\ 
 Constant & $-$74.486$^{***}$ & $-$69.443$^{***}$ & $-$81.732$^{***}$ & $-$29.257$^{***}$ & $-$31.468$^{***}$ & $-$47.620$^{***}$ & $-$17.262$^{***}$ & $-$35.574$^{***}$ & $-$52.757$^{***}$ & $-$33.680$^{***}$ \\ 
  & (4.891) & (3.771) & (4.640) & (2.666) & (1.794) & (4.026) & (3.271) & (4.281) & (3.857) & (2.368) \\ 
  & & & & & & & & & & \\ 
\hline \\[-1.8ex] 
Observations & 2,240 & 2,965 & 2,240 & 2,965 & 2,965 & 2,965 & 2,240 & 2,965 & 2,963 & 2,963 \\ 
R$^{2}$ & 0.135 & 0.324 & 0.427 & 0.499 & 0.020 & 0.084 & 0.056 & 0.063 & 0.120 & 0.062 \\ 
Adjusted R$^{2}$ & 0.132 & 0.323 & 0.424 & 0.498 & 0.019 & 0.082 & 0.054 & 0.062 & 0.118 & 0.060 \\ 
Residual Std. Error & 13.331 (df = 2231) & 13.441 (df = 2959) & 11.705 (df = 2228) & 8.962 (df = 2958) & 6.939 (df = 2961) & 14.919 (df = 2960) & 9.888 (df = 2234) & 15.481 (df = 2960) & 13.698 (df = 2956) & 8.010 (df = 2955) \\ 
F Statistic & 43.602$^{***}$ (df = 8; 2231) & 283.994$^{***}$ (df = 5; 2959) & 150.867$^{***}$ (df = 11; 2228) & 490.157$^{***}$ (df = 6; 2958) & 20.278$^{***}$ (df = 3; 2961) & 67.429$^{***}$ (df = 4; 2960) & 26.536$^{***}$ (df = 5; 2234) & 50.102$^{***}$ (df = 4; 2960) & 67.163$^{***}$ (df = 6; 2956) & 27.977$^{***}$ (df = 7; 2955) \\ 
\hline 
\hline \\[-1.8ex] 
\textit{Note:}  & \multicolumn{10}{r}{$^{*}$p$<$0.1; $^{**}$p$<$0.05; $^{***}$p$<$0.01} \\ 
\end{tabular}
}
\end{table}
\end{landscape}


\begin{landscape}
\begin{table}
[!htbp] \centering 
\caption{Instrumented regression with population density and expensive-food}
\resizebox{\columnwidth}{!}{
\begin{tabular}{@{\extracolsep{5pt}}lcccccccccc} 
\\[-1.8ex]\hline 
\hline \\[-1.8ex] 
 & \multicolumn{10}{c}{\textit{Dependent variable:}} \\ 
\cline{2-11} 
\\[-1.8ex] & lnvis & lndseducexpense & lnvis & lndshouserent & \multicolumn{6}{c}{lnvis} \\
\\[-1.8ex] & carpetsrugs(1) & education(2) & electricity(3) & houserent(4) & personalitemsrepair(5) & personalprods(6) & skincream(7) & funeral(8) & marriage(9) & hobbyequipment(10)\\ 
\hline \\[-1.8ex] 
 lnpinc & 5.814$^{***}$ & 4.758$^{***}$ & 10.958$^{***}$ & 0.635 & 0.988$^{***}$ & 3.216$^{***}$ & 0.982 & 1.828$^{**}$ & 1.465$^{*}$ & 2.074$^{***}$ \\ 
  & (1.149) & (1.010) & (1.512) & (0.425) & (0.302) & (0.565) & (0.612) & (0.902) & (0.781) & (0.705) \\ 
  & & & & & & & & & & \\ 
 age & $-$0.115$^{***}$ & 0.092$^{***}$ &  & $-$0.077$^{***}$ &  &  & $-$0.029 &  & $-$0.044$^{*}$ & $-$0.037$^{***}$ \\ 
  & (0.023) & (0.018) &  & (0.016) &  &  & (0.021) &  & (0.023) & (0.011) \\ 
  & & & & & & & & & & \\ 
 hsize & $-$0.607$^{***}$ & 2.011$^{***}$ & $-$1.248$^{***}$ &  &  & $-$0.518$^{***}$ & 0.383$^{***}$ &  &  &  \\ 
  & (0.173) & (0.152) & (0.209) &  &  & (0.140) & (0.102) &  &  &  \\ 
  & & & & & & & & & & \\ 
 housingstatus & 0.657$^{***}$ & $-$0.951$^{***}$ & 1.091$^{***}$ & 4.363$^{***}$ & 0.163 &  &  &  &  & 0.471$^{***}$ \\ 
  & (0.215) & (0.194) & (0.214) & (0.157) & (0.109) &  &  &  &  & (0.118) \\ 
  & & & & & & & & & & \\ 
 occupation\_rank &  &  & $-$0.660 &  &  &  &  & $-$1.030$^{**}$ &  &  \\ 
  &  &  & (0.475) &  &  &  &  & (0.485) &  &  \\ 
  & & & & & & & & & & \\ 
 isrural &  &  & $-$0.763 & $-$4.273$^{***}$ &  &  &  &  &  &  \\ 
  &  &  & (0.872) & (0.557) &  &  &  &  &  &  \\ 
  & & & & & & & & & & \\ 
 highest\_educ & $-$0.330$^{***}$ &  & 0.134 &  &  &  & $-$0.045 &  &  &  \\ 
  & (0.102) &  & (0.101) &  &  &  & (0.066) &  &  &  \\ 
  & & & & & & & & & & \\ 
 expensiveregion &  &  & 3.131$^{***}$ &  &  &  &  &  & $-$1.465 & $-$1.433$^{***}$ \\ 
  &  &  & (0.820) &  &  &  &  &  & (0.908) & (0.457) \\ 
  & & & & & & & & & & \\ 
 popdensity & $-$0.001$^{***}$ & $-$0.001$^{**}$ & 0.0002 &  & 0.0002 &  &  & 0.003$^{***}$ & 0.003$^{***}$ & 0.0003 \\ 
  & (0.0004) & (0.0005) & (0.0004) &  & (0.0002) &  &  & (0.0004) & (0.0004) & (0.0003) \\ 
  & & & & & & & & & & \\ 
 english & 2.659$^{***}$ &  & 3.306$^{***}$ &  &  &  &  &  & 2.879$^{***}$ & 0.243 \\ 
  & (0.983) &  & (0.945) &  &  &  &  &  & (0.996) & (0.736) \\ 
  & & & & & & & & & & \\ 
 years\_community &  &  & 0.094$^{***}$ &  &  & $-$0.077$^{***}$ & $-$0.054$^{***}$ &  &  &  \\ 
  &  &  & (0.022) &  &  & (0.020) & (0.018) &  &  &  \\ 
  & & & & & & & & & & \\ 
 roomsnum &  &  &  & $-$1.001$^{***}$ &  & 0.589$^{***}$ &  & 0.729$^{***}$ & 0.962$^{***}$ &  \\ 
  &  &  &  & (0.133) &  & (0.197) &  & (0.227) & (0.196) &  \\ 
  & & & & & & & & & & \\ 
 is\_resident & $-$1.655$^{**}$ &  & $-$1.151 & $-$2.425$^{***}$ &  &  &  &  &  & $-$0.324 \\ 
  & (0.730) &  & (0.854) & (0.492) &  &  &  &  &  & (0.410) \\ 
  & & & & & & & & & & \\ 
 Constant & $-$83.338$^{***}$ & $-$79.925$^{***}$ & $-$171.528$^{***}$ & $-$23.645$^{***}$ & $-$35.904$^{***}$ & $-$44.482$^{***}$ & $-$6.160 & $-$36.943$^{***}$ & $-$38.454$^{***}$ & $-$48.819$^{***}$ \\ 
  & (14.297) & (13.633) & (19.424) & (6.290) & (4.260) & (8.068) & (7.528) & (11.996) & (10.595) & (10.115) \\ 
  & & & & & & & & & & \\ 
\hline \\[-1.8ex] 
Observations & 2,240 & 2,965 & 2,240 & 2,240 & 2,240 & 2,240 & 2,240 & 2,240 & 2,240 & 2,963 \\ 
R$^{2}$ & 0.134 & 0.322 & 0.318 & 0.496 & 0.020 & 0.069 & 0.051 & 0.054 & 0.106 & 0.048 \\ 
Adjusted R$^{2}$ & 0.131 & 0.321 & 0.315 & 0.495 & 0.019 & 0.067 & 0.048 & 0.053 & 0.104 & 0.046 \\ 
Residual Std. Error & 13.343 (df = 2231) & 13.460 (df = 2959) & 12.765 (df = 2228) & 9.550 (df = 2233) & 7.418 (df = 2236) & 14.768 (df = 2235) & 9.917 (df = 2234) & 15.652 (df = 2235) & 14.160 (df = 2233) & 8.070 (df = 2955) \\ 
\hline 
\hline \\[-1.8ex] 
\textit{Note:}  & \multicolumn{10}{r}{$^{*}$p$<$0.1; $^{**}$p$<$0.05; $^{***}$p$<$0.01} \\ 
\end{tabular}
}
\end{table}
\end{landscape}


\bibliographystyle{plain}
\bibliography{C:/local_files/research/africa_attitudes}

\end{document}
